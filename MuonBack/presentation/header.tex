\documentclass[aspectratio=1610, professionalfonts, 13pt]{beamer}

% Lade das TU Dortund Theme von Max Nöthe
\usefonttheme[onlymath]{serif}
\usetheme[showtotalframes]{tudo}

% Lade richtiges Sprachpaket
\ifluatex
    \usepackage{polyglossia}
    \setmainlanguage{german}
\else
    \ifxetex
        \usepackage{polyglossia}
        \setmainlanguage{german}
    \else
        \usepackage[german]{babel}
    \fi
\fi

% Lade wichtige Mathematikpakete
\usepackage{amsmath}
\usepackage{amssymb}
\usepackage{mathtools}
\usepackage{cancel}
\usepackage[
  locale=DE,                   % deutsche Einstellungen
  separate-uncertainty=true,   % Immer Fehler mit \pm
  per-mode=symbol-or-fraction, % m/s im Text, sonst Brüche
]{siunitx}
\usepackage[absolute,overlay]{textpos}
\usepackage{framed}
\usepackage{multicol}
\usepackage{setspace}
\usepackage{graphicx}
\usepackage{booktabs}
\usepackage{caption}
\usepackage{appendixnumberbeamer}
\usepackage{tikz}
\usepackage[export]{adjustbox}


% Lade Paket zur Nutzung von Schleifen
\usepackage{forloop}

% ------------------------- Präsentationsinformationen -------------------------

% Titel:
\title{\textbf{Summerstudent project 2017}}
\subtitle{SHiP}

% Autoren:
\author{Kevin Sedlaczek}

% Datum:
\date{July to September 2017}

% Lehrstuhl/Fakultät:
\institute[CERN]{Search for Hidden Particles}

% Titelgrafik:
\titlegraphic{\includegraphics[width=0.25\textwidth]{CERN-logo.jpg}\hfill\includegraphics[width=0.25\textwidth]{SHiP-Full_Black_0.png}}

% ------------------------------------------------------------------------------
